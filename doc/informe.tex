\documentclass[11pt,a4paper]{article}
\usepackage[latin1]{inputenc}
\usepackage[spanish]{babel}
\usepackage{amsmath}
\usepackage{amssymb}
\usepackage{mathrsfs}
\usepackage{graphicx}
\usepackage{fancyvrb}
\addtolength{\textwidth}{2cm}
\addtolength{\marginparwidth}{-10cm}
\addtolength{\oddsidemargin}{-1cm}

\addtolength{\textwidth}{3cm}
\addtolength{\hoffset}{-1.5cm}
\addtolength{\textheight}{2cm}
\addtolength{\voffset}{-0.5cm}

% Inserts an horizontal line.
\newcommand{\Hrule}{\rule{\linewidth}{0.6mm}}

\fvset{frame=single}

%%%%%%%%%%%%%%%%%%%%%%%%%%%%%%%%%%%%%%%%%%%%%


\begin{document}


% Portada.

\begin{titlepage}

\begin{center}

\Hrule \\[0.4cm]
{\Huge \bfseries Proxy HTTP 1.1}\\[0.3cm]
\LARGE{Informe de desarrollo}
\Hrule \\[0.4cm]

\end{center}

\vfill

\begin{center}

\Large{Mat�as Ezequiel Colotto\\}
\Large{Mar�a Eugenia Cura\\}
\Large{Santiago Jos� Samra\\}
\Large{Jorge Ezequiel Scaruli\\}
\vspace{3cm}

\large{Protocolos de Comunicaci�n}\\
\large{2010}

\end{center}

\end{titlepage}


% Documento

\tableofcontents

\clearpage

\section{Protocolos desarrollados}

El proyecto se realiz� completa y �nicamente sobre el protocolo HTTP 1.1
especificado por la RFC 2616\footnote{http://tools.ietf.org/html/rfc2616}. Por
lo tanto, no se desarrollaron ni implementaron protocolos propios.

\section{Problemas encontrados}

Durante el desarrollo se encontraron gran cantidad de problemas, que debieron
ser solucionados para lograr un producto s�lido. Se mencionan a continuaci�n
algunos de ellos:

\subsection{Decisi�n sobre el uso de threads o E/S asincr�nica}

\subsection{Interpretaci�n y parseo de mensajes HTTP}

\subsection{Compatibilidad con HTTP 1.0}

\subsection{Conexiones persistentes}

\subsection{Proxy transparente}



\section{Limitaciones}

El proxy desarrollado tiene las siguientes limitaciones:

\begin{itemize}
  \item No soporta \emph{pipelining} de requests al servidor. Por ello, si bien
  se reusan las conexiones a los servidores cuando llegan distintos requests, la
  eficiencia de ello deja mucho que desear. De hecho, funciona m�s r�pidamente
  si las conexiones a los servidores no se reusan, cre�ndose una nueva cada vez
  que se realiza un request.
  \item No soporta todos los m�todos HTTP, solo GET, HEAD y POST; con lo cual
  hay ciertas p�ginas que no funcionan del todo bien porque usan otros m�todos 
  (por ejemplo, el chat de GMail, que usa CONNECT).
\end{itemize}

\section{Posibles extensiones}

\section{Conclusi�n}

\section{Ejemplos de testeo}

\section{Gu�a de instalaci�n}

\section{Gu�a de configuraci�n}

\section{Ejemplos de configuraci�n}

\section{Documento de dise�o}

\end{document}